% Options for packages loaded elsewhere
\PassOptionsToPackage{unicode}{hyperref}
\PassOptionsToPackage{hyphens}{url}
%
\documentclass[
]{article}
\usepackage{lmodern}
\usepackage{setspace}
\usepackage{amssymb,amsmath}
\usepackage{ifxetex,ifluatex}
\ifnum 0\ifxetex 1\fi\ifluatex 1\fi=0 % if pdftex
  \usepackage[T1]{fontenc}
  \usepackage[utf8]{inputenc}
  \usepackage{textcomp} % provide euro and other symbols
\else % if luatex or xetex
  \usepackage{unicode-math}
  \defaultfontfeatures{Scale=MatchLowercase}
  \defaultfontfeatures[\rmfamily]{Ligatures=TeX,Scale=1}
\fi
% Use upquote if available, for straight quotes in verbatim environments
\IfFileExists{upquote.sty}{\usepackage{upquote}}{}
\IfFileExists{microtype.sty}{% use microtype if available
  \usepackage[]{microtype}
  \UseMicrotypeSet[protrusion]{basicmath} % disable protrusion for tt fonts
}{}
\makeatletter
\@ifundefined{KOMAClassName}{% if non-KOMA class
  \IfFileExists{parskip.sty}{%
    \usepackage{parskip}
  }{% else
    \setlength{\parindent}{0pt}
    \setlength{\parskip}{6pt plus 2pt minus 1pt}}
}{% if KOMA class
  \KOMAoptions{parskip=half}}
\makeatother
\usepackage{xcolor}
\IfFileExists{xurl.sty}{\usepackage{xurl}}{} % add URL line breaks if available
\IfFileExists{bookmark.sty}{\usepackage{bookmark}}{\usepackage{hyperref}}
\hypersetup{
  pdftitle={Building a genetic linkage map of an autotetraploid population using MAPpoly},
  pdfauthor={Marcelo Mollinari, Gabriel Gesteira, Guilherme Pereira, A Augusto Garcia, Zhao-Bang Zeng},
  hidelinks,
  pdfcreator={LaTeX via pandoc}}
\urlstyle{same} % disable monospaced font for URLs
\usepackage[margin=1in]{geometry}
\usepackage{color}
\usepackage{fancyvrb}
\newcommand{\VerbBar}{|}
\newcommand{\VERB}{\Verb[commandchars=\\\{\}]}
\DefineVerbatimEnvironment{Highlighting}{Verbatim}{commandchars=\\\{\}}
% Add ',fontsize=\small' for more characters per line
\newenvironment{Shaded}{}{}
\newcommand{\AlertTok}[1]{\textcolor[rgb]{1.00,0.00,0.00}{\textbf{#1}}}
\newcommand{\AnnotationTok}[1]{\textcolor[rgb]{0.38,0.63,0.69}{\textbf{\textit{#1}}}}
\newcommand{\AttributeTok}[1]{\textcolor[rgb]{0.49,0.56,0.16}{#1}}
\newcommand{\BaseNTok}[1]{\textcolor[rgb]{0.25,0.63,0.44}{#1}}
\newcommand{\BuiltInTok}[1]{#1}
\newcommand{\CharTok}[1]{\textcolor[rgb]{0.25,0.44,0.63}{#1}}
\newcommand{\CommentTok}[1]{\textcolor[rgb]{0.38,0.63,0.69}{\textit{#1}}}
\newcommand{\CommentVarTok}[1]{\textcolor[rgb]{0.38,0.63,0.69}{\textbf{\textit{#1}}}}
\newcommand{\ConstantTok}[1]{\textcolor[rgb]{0.53,0.00,0.00}{#1}}
\newcommand{\ControlFlowTok}[1]{\textcolor[rgb]{0.00,0.44,0.13}{\textbf{#1}}}
\newcommand{\DataTypeTok}[1]{\textcolor[rgb]{0.56,0.13,0.00}{#1}}
\newcommand{\DecValTok}[1]{\textcolor[rgb]{0.25,0.63,0.44}{#1}}
\newcommand{\DocumentationTok}[1]{\textcolor[rgb]{0.73,0.13,0.13}{\textit{#1}}}
\newcommand{\ErrorTok}[1]{\textcolor[rgb]{1.00,0.00,0.00}{\textbf{#1}}}
\newcommand{\ExtensionTok}[1]{#1}
\newcommand{\FloatTok}[1]{\textcolor[rgb]{0.25,0.63,0.44}{#1}}
\newcommand{\FunctionTok}[1]{\textcolor[rgb]{0.02,0.16,0.49}{#1}}
\newcommand{\ImportTok}[1]{#1}
\newcommand{\InformationTok}[1]{\textcolor[rgb]{0.38,0.63,0.69}{\textbf{\textit{#1}}}}
\newcommand{\KeywordTok}[1]{\textcolor[rgb]{0.00,0.44,0.13}{\textbf{#1}}}
\newcommand{\NormalTok}[1]{#1}
\newcommand{\OperatorTok}[1]{\textcolor[rgb]{0.40,0.40,0.40}{#1}}
\newcommand{\OtherTok}[1]{\textcolor[rgb]{0.00,0.44,0.13}{#1}}
\newcommand{\PreprocessorTok}[1]{\textcolor[rgb]{0.74,0.48,0.00}{#1}}
\newcommand{\RegionMarkerTok}[1]{#1}
\newcommand{\SpecialCharTok}[1]{\textcolor[rgb]{0.25,0.44,0.63}{#1}}
\newcommand{\SpecialStringTok}[1]{\textcolor[rgb]{0.73,0.40,0.53}{#1}}
\newcommand{\StringTok}[1]{\textcolor[rgb]{0.25,0.44,0.63}{#1}}
\newcommand{\VariableTok}[1]{\textcolor[rgb]{0.10,0.09,0.49}{#1}}
\newcommand{\VerbatimStringTok}[1]{\textcolor[rgb]{0.25,0.44,0.63}{#1}}
\newcommand{\WarningTok}[1]{\textcolor[rgb]{0.38,0.63,0.69}{\textbf{\textit{#1}}}}
\usepackage{longtable,booktabs}
% Correct order of tables after \paragraph or \subparagraph
\usepackage{etoolbox}
\makeatletter
\patchcmd\longtable{\par}{\if@noskipsec\mbox{}\fi\par}{}{}
\makeatother
% Allow footnotes in longtable head/foot
\IfFileExists{footnotehyper.sty}{\usepackage{footnotehyper}}{\usepackage{footnote}}
\makesavenoteenv{longtable}
\usepackage{graphicx,grffile}
\makeatletter
\def\maxwidth{\ifdim\Gin@nat@width>\linewidth\linewidth\else\Gin@nat@width\fi}
\def\maxheight{\ifdim\Gin@nat@height>\textheight\textheight\else\Gin@nat@height\fi}
\makeatother
% Scale images if necessary, so that they will not overflow the page
% margins by default, and it is still possible to overwrite the defaults
% using explicit options in \includegraphics[width, height, ...]{}
\setkeys{Gin}{width=\maxwidth,height=\maxheight,keepaspectratio}
% Set default figure placement to htbp
\makeatletter
\def\fps@figure{htbp}
\makeatother
\setlength{\emergencystretch}{3em} % prevent overfull lines
\providecommand{\tightlist}{%
  \setlength{\itemsep}{0pt}\setlength{\parskip}{0pt}}
\setcounter{secnumdepth}{-\maxdimen} % remove section numbering

\title{Building a genetic linkage map of an autotetraploid population using
MAPpoly}
\author{Marcelo Mollinari, Gabriel Gesteira, Guilherme Pereira, A Augusto
Garcia, Zhao-Bang Zeng}
\date{2020-12-03}

\begin{document}
\maketitle

{
\setcounter{tocdepth}{4}
\tableofcontents
}
\setstretch{1.2}
\hypertarget{introduction}{%
\section{Introduction}\label{introduction}}

Linkage maps are essential tools to several genetic endeavours such as
quantitative trait loci (QTL) analysis, evolutionary studies, assessment
of colinearity in chromosome-wise scale between genomes, and study of
meiotic processes. The principle behind linkage map construction is
detecting recombinant events between genomic positions and summarizing
them into pairwise recombination fraction estimates. In diploids, the
assessment of such a phenomenon is relatively straightforward. After the
homolog duplication, sister chromatids pair up exchanging segments. The
presence of informative markers, e. g. single nucleotide polymorphisms
(SNPs) enable straightforward computation of the recombination fraction
between pairs of genomic positions by comparing the chromosome
constitution of parents and offspring. If the order of markers is
unknown, it can be obtained using the pairwise recombination fractions
in conjunction with optimization algorithms.

In polyploids (species with more than two sets of homolog chromosomes,
or homologs), the construction of such maps is quite challenging. While
in diploids a biallelic marker can be fully informative, in polyploids
they allow accounting for proportions of biallelic dosages. In order to
recover the multiallelic information present in polyploid species, we
need to account for recombination frequencies, estimate phase
configurations and reconstruct the haplotypes of both parents and
individuals in the population. As the ploidy level increases, the joint
computation of genomic regions becomes intensive, and other approaches
such as dimension reduction need to be applied.

\texttt{MAPpoly} is an R package fully capable of building genetic
linkage maps for biparental populations in polyploid species with ploidy
level up to 8x. This package is part of the
\href{https://sweetpotatogenomics.cals.ncsu.edu/}{Genomic Tools for
Sweetpotato Improvement} (GT4SP), funded by
\href{https://www.gatesfoundation.org/}{Bill and Melinda Gates
Foundation}. All of the procedures presented in this document are
detailed in Mollinari and Garcia (2019) and ({\textbf{???}}) 2020. The
results obtained with \texttt{MAPpoly} can be readily used for QTL
mapping with the
\href{https://github.com/guilherme-pereira/QTLpoly}{QTLpoly package},
which implements the procedures proposed by Pereira et al. (2020).

Some advantages of \texttt{MAPpoly} are:

\begin{itemize}
\tightlist
\item
  Can handle multiple dataset types
\item
  Can handle thousands of markers
\item
  Does not depend on single-dose markers (SDM) to build the map
\item
  Incorporates genomic information
\item
  Explores multipoint information (through Hidden Markov Chain Model)
\item
  Can handle high ploidies: up to 8x when using HMM, and up to 12x when
  using the two-point approach
\item
  Can reconstruct haplotypes for parents and all individuals in the
  population
\item
  Recovers the multiallelic nature of polyploid genomes
\item
  Detects occurrence, location and frequency of multivalent pairing
  during meiosis
\item
  Robust enough to build genetic linkage maps with multiallelic markers
  (when available)
\end{itemize}

\hypertarget{mappoly-installation}{%
\subsection{\texorpdfstring{\texttt{MAPpoly}
installation}{MAPpoly installation}}\label{mappoly-installation}}

\hypertarget{from-cran-stable-version}{%
\subsection{From CRAN (stable version)}\label{from-cran-stable-version}}

To install MAPpoly from the The Comprehensive R Archive Network (CRAN)
use

\begin{Shaded}
\begin{Highlighting}[]
\KeywordTok{install.packages}\NormalTok{(}\StringTok{"mappoly"}\NormalTok{)}
\end{Highlighting}
\end{Shaded}

\hypertarget{from-github-development-version}{%
\subsection{From GitHub (development
version)}\label{from-github-development-version}}

You can install the development version from Git Hub. Within R, you need
to install \texttt{devtools}:

\begin{Shaded}
\begin{Highlighting}[]
\KeywordTok{install.packages}\NormalTok{(}\StringTok{"devtools"}\NormalTok{)}
\end{Highlighting}
\end{Shaded}

If you are using Windows, you must install the the latest recommended
version of
\href{https://cran.r-project.org/bin/windows/Rtools/}{Rtools}.

To install MAPpoly from Git Hub use

\begin{Shaded}
\begin{Highlighting}[]
\NormalTok{devtools}\OperatorTok{::}\KeywordTok{install_github}\NormalTok{(}\StringTok{"mmollina/mappoly"}\NormalTok{, }\DataTypeTok{dependencies=}\OtherTok{TRUE}\NormalTok{)}
\end{Highlighting}
\end{Shaded}

\hypertarget{loading-datasets}{%
\section{Loading datasets}\label{loading-datasets}}

In its current version, MAPpoly can handle the following types of
datasets:

\begin{enumerate}
\def\labelenumi{\arabic{enumi}.}
\tightlist
\item
  CSV files
\item
  MAPpoly files

  \begin{itemize}
  \tightlist
  \item
    Dosage based
  \item
    Probability based
  \end{itemize}
\item
  \href{https://CRAN.R-project.org/package=fitPoly}{fitPoly} files
\item
  VCF files
\end{enumerate}

MAPpoly also is capable of importing objects generated by the following
R packages

\begin{enumerate}
\def\labelenumi{\arabic{enumi}.}
\tightlist
\item
  \href{https://CRAN.R-project.org/package=updog}{updog}
\item
  \href{https://CRAN.R-project.org/package=polyRAD}{polyRAD}
\item
  \href{https://CRAN.R-project.org/package=polymapR}{polymapR}

  \begin{itemize}
  \tightlist
  \item
    Datasets
  \item
    Maps
  \end{itemize}
\end{enumerate}

Both CSV and \texttt{MAPpoly} datasets are sensible formatting errors,
such as additional spaces, commas and wrong encoding (non-UTF-8). If you
have any trouble, please double check your files before submitting an
issue. Detailed steps of all supported files are given on the sections
below. Also, a considerable proportion of redundant markers is expected
on large datasets. Since markers having the same information do not
provide any advantage during the mapping process, redundant ones may be
removed from the dataset in order to reduce computational effort. All
reading functions share the \texttt{elim.redundant} argument that
automatically identifies and removes redundant markers during the
analysis. The redundant markers can always be recovered and exported
when the final map is ready.

\hypertarget{reading-csv-files}{%
\subsection{Reading CSV files}\label{reading-csv-files}}

The preparation of a CSV file for MAPpoly is relatively straightforward.
It can be done in Microsoft Excel or any other spreadsheet software of
your preference. In this file, each line comprehends a marker and each
column comprehends information about the marker. In its current version,
\texttt{MAPpoly} can handle .csv files with allelic dosage data.

The first line of the CSV file should contain headers for all columns.
The first five columns should include the following information: marker
name, dosage for both parents (one column for each), a sequence number
(e.g.~a chromosome number, if available) and a sequence position
(e.g.~the marker position within the chromosome, if available). In
addition to these five headers, you should include the name of all
individuals in the population. From the second line onwards, all columns
should contain its values, including allelic dosages for all
individuals. Missing or absent values should be represented by NA.

NOTE: If genomic information is not available, the `sequence' and
`sequence position' columns should be filled with NA's.

Example:

\textbf{Important note: avoid spaces in .csv files.} As mentioned above,
please double check your datasets for extra spaces, commas, dots and
encoding. Your CSV file should be encoded using UTF-8.

You can read CSV files with the \texttt{read\_geno\_csv} function:

\begin{Shaded}
\begin{Highlighting}[]
\NormalTok{ft=}\StringTok{"https://raw.githubusercontent.com/mmollina/SCRI/main/data/B2721_dose.csv"}
\NormalTok{tempfl <-}\StringTok{ }\KeywordTok{tempfile}\NormalTok{()}
\KeywordTok{download.file}\NormalTok{(ft, }\DataTypeTok{destfile =}\NormalTok{ tempfl)}
\NormalTok{dat.dose.csv <-}\StringTok{ }\KeywordTok{read_geno_csv}\NormalTok{(}\DataTypeTok{file.in  =}\NormalTok{ tempfl, }\DataTypeTok{ploidy =} \DecValTok{4}\NormalTok{)}
\end{Highlighting}
\end{Shaded}

\begin{Shaded}
\begin{Highlighting}[]
\NormalTok{dat.dose.csv}
\end{Highlighting}
\end{Shaded}

In addition to the CSV file path, you should indicate the ploidy level
using the \texttt{ploidy} argument. This function automatically excludes
uninformative markers. It also performs chi-square tests for all
markers, considering the expected segregation patterns under Mendelian
inheritance, random chromosome pairing and no double reduction. You can
optionally use the \texttt{filter.non.conforming} logical argument
(default = TRUE), which excludes non-expected genotypes under these
assumptions. However, keep in mind that the functions in MAPpoly do not
support double-reduction in its current version, and this can cause the
software to abort.

\hypertarget{reading-mappoly-files}{%
\subsection{Reading MAPpoly files}\label{reading-mappoly-files}}

Besides CSV and VCF files, MAPpoly can also handle two dataset types
that follow the same format: (1) a genotype-based file (with allelic
dosages) and (2) probability-based file. Both are text files with the
same header, but with different genotype table formats.

For both files, the header should contain: ploidy level, number of
individuals (nind), number of markers (nmrk), marker names (mrknames),
individual names (indnames), allele dosages for parent 1 (dosageP),
allele dosages for parent 2 (dosageQ), sequence/chromosome information
(seq), position of each marker (seqpos), number of phenotypic traits
(nphen) and the phenotypic data (pheno) if available. The header should
be organized according to this example:

\begin{verbatim}
ploidy 4
nind 3
nmrk 5
mrknames M1 M2 M3 M4 M5
indnames Ind1 Ind2 Ind3
dosageP 0 2 0 0 3
dosageQ 1 2 1 1 3
seq 1 1 2 2 3
seqpos 100 200 50 150 80
nphen 0
pheno-----------------------
geno------------------------
\end{verbatim}

For more information about MAPpoly file format, please see
\texttt{?read\_geno} and \texttt{?read\_geno\_prob} documentation from
\texttt{MAPpoly} package.

\hypertarget{using-read_geno}{%
\subsubsection{\texorpdfstring{Using
\texttt{read\_geno}}{Using read\_geno}}\label{using-read_geno}}

The header should be followed by a table containing the genotypes
(allele dosages) for each marker (rows) and for each individual
(columns), as follows:

\begin{longtable}[]{@{}lccc@{}}
\toprule
& Individual 1 & Individual 2 & Individual 3\tabularnewline
\midrule
\endhead
Marker 1 & 1 & 0 & 0\tabularnewline
Marker 2 & 3 & 0 & 2\tabularnewline
Marker 3 & 1 & 0 & 0\tabularnewline
Marker 4 & 1 & 0 & 0\tabularnewline
Marker 5 & 3 & 4 & 4\tabularnewline
\bottomrule
\end{longtable}

The final file should look like the example below:

\begin{verbatim}
ploidy 4
nind 3
nmrk 5
mrknames M1 M2 M3 M4 M5
indnames Ind1 Ind2 Ind3
dosageP 0 2 0 0 3
dosageQ 1 2 1 1 3
seq 1 1 2 2 3
seqpos 100 200 50 150 80
nphen 0
pheno-----------------------
geno------------------------
1 0 0
3 0 2
1 0 0
1 0 0
3 4 4
\end{verbatim}

Then, use the \texttt{read\_geno} function to read your file:

\begin{Shaded}
\begin{Highlighting}[]
\NormalTok{fl =}\StringTok{ "https://raw.githubusercontent.com/mmollina/MAPpoly_vignettes/master/data/SolCAP_dosage"}
\NormalTok{tempfl <-}\StringTok{ }\KeywordTok{tempfile}\NormalTok{()}
\KeywordTok{download.file}\NormalTok{(fl, }\DataTypeTok{destfile =}\NormalTok{ tempfl)}
\NormalTok{dat.dose.mpl <-}\StringTok{ }\KeywordTok{read_geno}\NormalTok{(}\DataTypeTok{file.in  =}\NormalTok{ tempfl, }\DataTypeTok{elim.redundant =} \OtherTok{TRUE}\NormalTok{)}
\end{Highlighting}
\end{Shaded}

\begin{Shaded}
\begin{Highlighting}[]
\NormalTok{dat.dose.mpl}
\end{Highlighting}
\end{Shaded}

\hypertarget{using-read_geno_prob}{%
\subsubsection{\texorpdfstring{Using
\texttt{read\_geno\_prob}}{Using read\_geno\_prob}}\label{using-read_geno_prob}}

Following the same header described before, \texttt{read\_geno\_prob}
reads a table containing the probability distribution for each
combination of marker \(\times\) individual. Each line on this table
represents the combination of one marker with one individual, and its
respective probabilities of having each possible allele dosage. The
first two columns represent the marker and the individual, respectively,
and the remaining elements represent the probability associated with
each one of the possible dosages, as follows:

\begin{longtable}[]{@{}lcccccc@{}}
\toprule
Marker & Individual & \(p(d=0)\) & \(p(d=1)\) & \(p(d=2)\) & \(p(d=3)\)
& \(p(d=4)\)\tabularnewline
\midrule
\endhead
M1 & Ind1 & 0.5 & 0.5 & 0.0 & 0.0 & 0.0\tabularnewline
M2 & Ind1 & 0.0 & 1.0 & 0.0 & 0.0 & 0.0\tabularnewline
M3 & Ind1 & 0.3 & 0.7 & 0.0 & 0.0 & 0.0\tabularnewline
M4 & Ind1 & 0.5 & 0.5 & 0.0 & 0.0 & 0.0\tabularnewline
M5 & Ind1 & 0.0 & 0.0 & 0.0 & 0.9 & 0.1\tabularnewline
M1 & Ind2 & 1.0 & 0.0 & 0.0 & 0.0 & 0.0\tabularnewline
M2 & Ind2 & 0.2 & 0.5 & 0.3 & 0.0 & 0.0\tabularnewline
M3 & Ind2 & 0.9 & 0.1 & 0.0 & 0.0 & 0.0\tabularnewline
M4 & Ind2 & 0.9 & 0.1 & 0.0 & 0.0 & 0.0\tabularnewline
M5 & Ind2 & 0.0 & 0.0 & 0.0 & 0.2 & 0.8\tabularnewline
M1 & Ind3 & 0.2 & 0.8 & 0.0 & 0.0 & 0.0\tabularnewline
M2 & Ind3 & 0.4 & 0.6 & 0.0 & 0.0 & 0.0\tabularnewline
M3 & Ind3 & 1.0 & 0.0 & 0.0 & 0.0 & 0.0\tabularnewline
M4 & Ind3 & 0.0 & 0.1 & 0.9 & 0.0 & 0.0\tabularnewline
M5 & Ind3 & 0.1 & 0.9 & 0.0 & 0.0 & 0.0\tabularnewline
\bottomrule
\end{longtable}

Notice that each marker \(\times\) individual combination have \(m+1\)
associated probabilities, being \(m\) the ploidy level and \(m+1\) the
number of possible allele dosages. Also, each line must sum to 1. The
final file (header + table) should look like the following example:

\begin{verbatim}
ploidy 4
nind 3
nmrk 5
mrknames M1 M2 M3 M4 M5
indnames Ind1 Ind2 Ind3
dosageP 0 2 0 0 3
dosageQ 1 2 1 1 3
seq 1 1 2 2 3
seqpos 100 200 50 150 80
nphen 0
pheno-----------------------
geno------------------------
M1 Ind1 0.5 0.5 0.0 0.0 0.0
M2 Ind1 0.0 1.0 0.0 0.0 0.0
M3 Ind1 0.3 0.7 0.0 0.0 0.0
M4 Ind1 0.5 0.5 0.0 0.0 0.0
M5 Ind1 0.0 0.0 0.0 0.9 0.1
M1 Ind2 1.0 0.0 0.0 0.0 0.0
M2 Ind2 0.2 0.5 0.3 0.0 0.0
M3 Ind2 0.9 0.1 0.0 0.0 0.0
M4 Ind2 0.9 0.1 0.0 0.0 0.0
M5 Ind2 0.0 0.0 0.0 0.2 0.8
M1 Ind3 0.2 0.8 0.0 0.0 0.0
M2 Ind3 0.4 0.6 0.0 0.0 0.0
M3 Ind3 1.0 0.0 0.0 0.0 0.0
M4 Ind3 0.0 0.1 0.9 0.0 0.0
M5 Ind3 0.1 0.9 0.0 0.0 0.0
\end{verbatim}

To read the dataset, one should use:

\begin{Shaded}
\begin{Highlighting}[]
\NormalTok{ft=}\StringTok{"https://raw.githubusercontent.com/mmollina/MAPpoly_vignettes/master/data/SolCAP"}
\NormalTok{tempfl <-}\StringTok{ }\KeywordTok{tempfile}\NormalTok{()}
\KeywordTok{download.file}\NormalTok{(ft, }\DataTypeTok{destfile =}\NormalTok{ tempfl)}
\NormalTok{dat.prob.mpl <-}\StringTok{ }\KeywordTok{read_geno_prob}\NormalTok{(}\DataTypeTok{file.in  =}\NormalTok{ tempfl, }\DataTypeTok{prob.thres =} \FloatTok{0.95}\NormalTok{, }\DataTypeTok{elim.redundant =} \OtherTok{TRUE}\NormalTok{)}
\end{Highlighting}
\end{Shaded}

\begin{Shaded}
\begin{Highlighting}[]
\NormalTok{dat.prob.mpl}
\end{Highlighting}
\end{Shaded}

\textbf{Important note:} as this type of file contains the probability
distribution of all possible dosages, it will take longer to read.

This function automatically excludes monomorphic markers, keeping only
informative ones. You can define the minimum probability value necessary
to call a dosage using the \texttt{prob.thres} argument. If the higher
probability for a marker \(\times\) individual passes this threshold,
then its associated dosage is used. However, if none of the
probabilities reach this threshold, then its dosage is considered
missing (NA). This function also performs chi-square tests for all
markers, considering the expected segregation patterns under Mendelian
inheritance, random chromosome pairing and no double reduction. You can
optionally use the \texttt{filter.non.conforming} logical argument
(default = TRUE), which excludes non-expected genotypes under these
assumptions.

\hypertarget{read_vcf}{%
\subsection{Reading VCF files}\label{read_vcf}}

VCF files are less sensible to errors, because they are usually produced
by automated SNP calling pipelines and less susceptible to user edition.
\texttt{MAPpoly} can also handle VCF files of version 4.0 and higher
produced by the most common softwares, such as TASSEL, GATK, Stacks and
many others. As few of these softwares can handle poliploidy and
estimate genotypes correctly, you may use other softwares that are
dedicated to estimate the allele dosages. Briefly, these softwares model
the ratio between allele read counts (or intensity) for each marker
\(\times\) individual combination, and determines which is the most
probable allele dosage given the observed ratio and other \emph{a
priori} information. Examples of these softwares are
\href{http://statgen.esalq.usp.br/SuperMASSA/}{SuperMASSA},
\href{https://cran.r-project.org/web/packages/fitPoly/index.html}{fitPoly},
\href{https://potatobreeding.cals.wisc.edu/wp-content/uploads/sites/161/2017/08/ClusterCall_Download.zip}{ClusterCall},
\href{https://cloud.r-project.org/web/packages/updog/index.html}{updog},
\href{https://cran.r-project.org/web/packages/polyRAD/vignettes/polyRADtutorial.html}{PolyRAD}
and many others. After allele dosage estimation, your VCF file should
contain GT values like \textbf{1/1/1/0} (for an autotetraploid) rather
than \textbf{1/0}. Since \texttt{MAPpoly} uses allelic dosages (or their
probabilities) to build genetic maps, we strongly recommend that you use
one of these softwares to estimate allele dosages before building the
map. Both \texttt{PolyRAD} and \texttt{updog} have direct integration
with \texttt{MAPpoly}, as described in the next section.

To demonstrate the \texttt{read\_vcf} function, lets download an
autohexaploid sweetpotato VCF file from the MAPpoly's vignettes
repository on Github and read it:

\begin{Shaded}
\begin{Highlighting}[]
\KeywordTok{download.file}\NormalTok{(}\StringTok{"https://github.com/mmollina/MAPpoly_vignettes/raw/master/data/BT/sweetpotato_chr1.vcf.gz"}\NormalTok{, }\DataTypeTok{destfile =} \StringTok{'chr1.vcf.gz'}\NormalTok{)}
\NormalTok{dat.dose.vcf =}\StringTok{ }\KeywordTok{read_vcf}\NormalTok{(}\DataTypeTok{file =} \StringTok{'chr1.vcf.gz'}\NormalTok{, }\DataTypeTok{parent.1 =} \StringTok{"PARENT1"}\NormalTok{, }\DataTypeTok{parent.2 =} \StringTok{"PARENT2"}\NormalTok{)}
\end{Highlighting}
\end{Shaded}

\begin{Shaded}
\begin{Highlighting}[]
\NormalTok{dat.dose.vcf}
\end{Highlighting}
\end{Shaded}

Besides the path to your VCF file, you should indicate \texttt{parent.1}
and \texttt{parent.2} names. Please notice that their names must be
exactly the same strings that appear in your VCF file. The ploidy level
will be automatically detected, but you may indicate it using the
optional \texttt{ploidy} argument to let the function check for possible
errors. For species with variable ploidy levels (i.e.~sugarcane), please
indicate the desired ploidy level using the \texttt{ploidy} argument; if
absent, \texttt{MAPpoly} will use the smallest ploidy level detected.

This function also has options to filter out undesired markers or data
points, such as those with low depth or high proportion of missing data.
You can define the following filter arguments: set \texttt{min.av.depth}
to any integer level in order to remove markers that show average depth
below this value (default = 0); set \texttt{min.gt.depth} to any integer
level in order to remove data points that present depth below this value
(default = 0); set \texttt{max.missing} to any value between 0 and 1 in
order to remove markers that present missing data proportion above this
value (default = 1).

\texttt{read\_vcf} performs chi-square tests for all markers,
considering the expected segregation patterns under Mendelian
inheritance, random chromosome pairing and no double reduction. You can
optionally use the \texttt{filter.non.conforming} logical argument
(default = TRUE), which excludes non-expected genotypes under these
assumptions. The p-value threshold used by the segregation test can be
defined by the \texttt{thresh.line} argument.

Please notice that the returning object from \texttt{read\_vcf} has some
additional information: reference and alternative alleles (bases) for
each marker; and average depth of each marker. You can inspect all
marker depths using the following code as an example:

\begin{Shaded}
\begin{Highlighting}[]
\KeywordTok{library}\NormalTok{(ggplot2)}
\NormalTok{dosage_combs =}\StringTok{ }\KeywordTok{cbind}\NormalTok{(dat.dose.vcf}\OperatorTok{$}\NormalTok{dosage.p, dat.dose.vcf}\OperatorTok{$}\NormalTok{dosage.q)}
\NormalTok{dc_simplex =}\StringTok{ }\KeywordTok{apply}\NormalTok{(dosage_combs,}\DecValTok{1}\NormalTok{,}\ControlFlowTok{function}\NormalTok{(x) }\ControlFlowTok{if}\NormalTok{(}\KeywordTok{all}\NormalTok{(}\KeywordTok{c}\NormalTok{(}\DecValTok{0}\NormalTok{,}\DecValTok{1}\NormalTok{) }\OperatorTok\StringTok{ }\NormalTok{x) }\OperatorTok{|}\StringTok{ }\KeywordTok{all}\NormalTok{(}\KeywordTok{c}\NormalTok{(dat.dose.vcf}\OperatorTok{$}\NormalTok{m}\DecValTok{-1}\NormalTok{, dat.dose.vcf}\OperatorTok{$}\NormalTok{m) }\OperatorTok\StringTok{ }\NormalTok{x)) }\KeywordTok{return}\NormalTok{(}\OtherTok{TRUE}\NormalTok{) }\ControlFlowTok{else} \KeywordTok{return}\NormalTok{(}\OtherTok{FALSE}\NormalTok{))}
\NormalTok{dc_dsimplex =}\StringTok{ }\KeywordTok{apply}\NormalTok{(dosage_combs,}\DecValTok{1}\NormalTok{,}\ControlFlowTok{function}\NormalTok{(x) }\ControlFlowTok{if}\NormalTok{(}\KeywordTok{all}\NormalTok{(x }\OperatorTok{==}\StringTok{ }\KeywordTok{c}\NormalTok{(}\DecValTok{1}\NormalTok{,}\DecValTok{1}\NormalTok{)) }\OperatorTok{|}\StringTok{ }\KeywordTok{all}\NormalTok{(x }\OperatorTok{==}\StringTok{ }\KeywordTok{c}\NormalTok{(dat.dose.vcf}\OperatorTok{$}\NormalTok{m}\DecValTok{-1}\NormalTok{, dat.dose.vcf}\OperatorTok{$}\NormalTok{m}\DecValTok{-1}\NormalTok{))) }\KeywordTok{return}\NormalTok{(}\OtherTok{TRUE}\NormalTok{) }\ControlFlowTok{else} \KeywordTok{return}\NormalTok{(}\OtherTok{FALSE}\NormalTok{))}

\NormalTok{dc_simplex[}\KeywordTok{which}\NormalTok{(dc_simplex }\OperatorTok{==}\StringTok{ }\OtherTok{TRUE}\NormalTok{)] =}\StringTok{ "simplex"}
\NormalTok{dc_simplex[}\KeywordTok{which}\NormalTok{(dc_dsimplex }\OperatorTok{==}\StringTok{ }\OtherTok{TRUE}\NormalTok{)] =}\StringTok{ 'double simplex'}
\NormalTok{dc_simplex[}\KeywordTok{which}\NormalTok{(dc_simplex }\OperatorTok{==}\StringTok{ 'FALSE'}\NormalTok{)] =}\StringTok{ 'multiplex'}

\NormalTok{data_depths =}\StringTok{ }\KeywordTok{data.frame}\NormalTok{(}\StringTok{'Marker depths'}\NormalTok{ =}\StringTok{ }\NormalTok{dat.dose.vcf}\OperatorTok{$}\NormalTok{all.mrk.depth,}
                         \StringTok{'Depth classes'}\NormalTok{ =}\StringTok{ }\KeywordTok{findInterval}\NormalTok{(dat.dose.vcf}\OperatorTok{$}\NormalTok{all.mrk.depth, }\KeywordTok{c}\NormalTok{(}\DecValTok{200}\NormalTok{,}\DecValTok{300}\NormalTok{,}\DecValTok{400}\NormalTok{,}\DecValTok{500}\NormalTok{,}\DecValTok{600}\NormalTok{,}\DecValTok{50000}\NormalTok{)),}
                         \StringTok{'Dosage combinations'}\NormalTok{ =}\StringTok{ }\NormalTok{dc_simplex, }\DataTypeTok{check.names =}\NormalTok{ F)}

\KeywordTok{ggplot}\NormalTok{(data_depths, }\KeywordTok{aes}\NormalTok{(}\DataTypeTok{fill=}\StringTok{`}\DataTypeTok{Dosage combinations}\StringTok{`}\NormalTok{, }\DataTypeTok{x=}\StringTok{`}\DataTypeTok{Depth classes}\StringTok{`}\NormalTok{, }\DataTypeTok{y=}\StringTok{`}\DataTypeTok{Marker depths}\StringTok{`}\NormalTok{)) }\OperatorTok{+}
\StringTok{  }\KeywordTok{geom_bar}\NormalTok{(}\DataTypeTok{position =} \StringTok{'stack'}\NormalTok{, }\DataTypeTok{stat =} \StringTok{'identity'}\NormalTok{) }\OperatorTok{+}
\StringTok{  }\KeywordTok{scale_x_continuous}\NormalTok{(}\DataTypeTok{breaks=}\DecValTok{0}\OperatorTok{:}\DecValTok{5}\NormalTok{, }\DataTypeTok{labels=}\KeywordTok{c}\NormalTok{(}\StringTok{">200"}\NormalTok{,}\StringTok{"200-300"}\NormalTok{,}\StringTok{"300-400"}\NormalTok{,}\StringTok{"400-500"}\NormalTok{,}\StringTok{"500-600"}\NormalTok{, }\StringTok{">600"}\NormalTok{))}
\end{Highlighting}
\end{Shaded}

\hypertarget{importing-data-from-third-party-packages}{%
\subsection{Importing data from third party
packages}\label{importing-data-from-third-party-packages}}

As mentioned above, polyploid sequencing data must be used to estimate
allelic dosages prior to linkage map building. The R package
\texttt{PolyRAD} has its own function to export genotypes to the
\texttt{MAPpoly}'s genotype probability distribution format. One may use
the commands above to import from \texttt{PolyRAD}:

\begin{Shaded}
\begin{Highlighting}[]
\CommentTok{# load example dataset from polyRAD}
\KeywordTok{library}\NormalTok{(polyRAD)}
\KeywordTok{data}\NormalTok{(exampleRAD_mapping)}
\NormalTok{exampleRAD_mapping =}\StringTok{ }\KeywordTok{SetDonorParent}\NormalTok{(exampleRAD_mapping, }\StringTok{"parent1"}\NormalTok{)}
\NormalTok{exampleRAD_mapping =}\StringTok{ }\KeywordTok{SetRecurrentParent}\NormalTok{(exampleRAD_mapping, }\StringTok{"parent2"}\NormalTok{)}
\NormalTok{exampleRAD_mapping =}\StringTok{ }\KeywordTok{PipelineMapping2Parents}\NormalTok{(exampleRAD_mapping)}

\CommentTok{# export to MAPpoly}
\NormalTok{outfile2 =}\StringTok{ }\KeywordTok{tempfile}\NormalTok{()}
\KeywordTok{Export_MAPpoly}\NormalTok{(exampleRAD_mapping, }\DataTypeTok{file =}\NormalTok{ outfile2)}

\CommentTok{# Read in MAPpoly}
\NormalTok{mydata_polyrad =}\StringTok{ }\KeywordTok{read_geno_prob}\NormalTok{(outfile2)}
\end{Highlighting}
\end{Shaded}

\begin{Shaded}
\begin{Highlighting}[]
\NormalTok{mydata_polyrad}
\end{Highlighting}
\end{Shaded}

You can also use the \texttt{MAPpoly}'s function
\texttt{import\_from\_updog} to import any dataset generated by
\texttt{updog}'s function \texttt{multidog}, following the example
below:

\begin{Shaded}
\begin{Highlighting}[]
\CommentTok{# Load example dataset from updog}
\KeywordTok{library}\NormalTok{(updog)}
\KeywordTok{data}\NormalTok{(uitdewilligen)}
\NormalTok{mout =}\StringTok{ }\KeywordTok{multidog}\NormalTok{(}\DataTypeTok{refmat =} \KeywordTok{t}\NormalTok{(uitdewilligen}\OperatorTok{$}\NormalTok{refmat), }
                \DataTypeTok{sizemat =} \KeywordTok{t}\NormalTok{(uitdewilligen}\OperatorTok{$}\NormalTok{sizemat), }
                \DataTypeTok{ploidy =}\NormalTok{ uitdewilligen}\OperatorTok{$}\NormalTok{ploidy, }
                \DataTypeTok{model =} \StringTok{"f1"}\NormalTok{,}
                \DataTypeTok{p1_id =} \KeywordTok{colnames}\NormalTok{(}\KeywordTok{t}\NormalTok{(uitdewilligen}\OperatorTok{$}\NormalTok{sizemat))[}\DecValTok{1}\NormalTok{],}
                \DataTypeTok{p2_id =} \KeywordTok{colnames}\NormalTok{(}\KeywordTok{t}\NormalTok{(uitdewilligen}\OperatorTok{$}\NormalTok{sizemat))[}\DecValTok{2}\NormalTok{],}
                \DataTypeTok{nc =} \DecValTok{4}\NormalTok{)}
\NormalTok{mydata_updog =}\StringTok{ }\KeywordTok{import_from_updog}\NormalTok{(mout)}
\end{Highlighting}
\end{Shaded}

\begin{Shaded}
\begin{Highlighting}[]
\NormalTok{mydata_updog}
\end{Highlighting}
\end{Shaded}

Please notice that \texttt{updog} removes both sequence and sequence
position information that may be present in the VCF file. We highly
recommend that you use this information during the linkage map building,
when available.

\hypertarget{combining-multiple-datasets}{%
\subsection{Combining multiple
datasets}\label{combining-multiple-datasets}}

It is not rare to have two or more datasets regarding the same
population or individuals from different sources of molecular data, such
as SNP chips, GBS and/or microsatellites. \texttt{MAPpoly} can combine
datasets when individual names are the same, using the function
\texttt{merge\_datasets}. To demonstrate its functionality, lets
download two VCF files (autohexaploid sweetpotato) from the MAPpoly
vignettes repository on Github, and read them using \texttt{read\_vcf}
function:

\begin{Shaded}
\begin{Highlighting}[]
\CommentTok{# Downloading VCF files regarding chromosome 1 and 2}
\KeywordTok{download.file}\NormalTok{(}\StringTok{"https://github.com/mmollina/MAPpoly_vignettes/raw/master/data/BT/sweetpotato_chr1.vcf.gz"}\NormalTok{, }\DataTypeTok{destfile =} \StringTok{'chr1.vcf.gz'}\NormalTok{)}
\KeywordTok{download.file}\NormalTok{(}\StringTok{"https://github.com/mmollina/MAPpoly_vignettes/raw/master/data/BT/sweetpotato_chr2.vcf.gz"}\NormalTok{, }\DataTypeTok{destfile =} \StringTok{'chr2.vcf.gz'}\NormalTok{)}
\NormalTok{data1 =}\StringTok{ }\KeywordTok{read_vcf}\NormalTok{(}\DataTypeTok{file =} \StringTok{'chr1.vcf.gz'}\NormalTok{, }\DataTypeTok{parent.1 =} \StringTok{"PARENT1"}\NormalTok{, }\DataTypeTok{parent.2 =} \StringTok{"PARENT2"}\NormalTok{)}
\NormalTok{data2 =}\StringTok{ }\KeywordTok{read_vcf}\NormalTok{(}\DataTypeTok{file =} \StringTok{'chr2.vcf.gz'}\NormalTok{, }\DataTypeTok{parent.1 =} \StringTok{"PARENT1"}\NormalTok{, }\DataTypeTok{parent.2 =} \StringTok{"PARENT2"}\NormalTok{)}
\end{Highlighting}
\end{Shaded}

As we can see, both have different markers for the same population:

\begin{Shaded}
\begin{Highlighting}[]
\CommentTok{# See datasets}
\KeywordTok{print}\NormalTok{(data1)}
\KeywordTok{print}\NormalTok{(data2)}
\end{Highlighting}
\end{Shaded}

Now lets merge them and see the output:

\begin{Shaded}
\begin{Highlighting}[]
\CommentTok{# Merge datasets}
\NormalTok{merged_data =}\StringTok{ }\KeywordTok{merge_datasets}\NormalTok{(data1, data2)}
\end{Highlighting}
\end{Shaded}

\begin{Shaded}
\begin{Highlighting}[]
\KeywordTok{print}\NormalTok{(merged_data)}
\end{Highlighting}
\end{Shaded}

Notice that all markers of both datasets were merged successfully, which
allows using just one (merged) dataset in the following steps of map
construction.

\hypertarget{exploratory-analysis}{%
\subsection{Exploratory Analysis}\label{exploratory-analysis}}

\hypertarget{whole-dataset}{%
\subsubsection{Whole dataset}\label{whole-dataset}}

For didatic purposes, we will keep using the tetraploid potato array
data (loaded using the examples above). We will construct a genetic map
of the B2721 population, which is a cross between two tetraploid potato
varieties: Atlantic and B1829-5. The population comprises 160 offsprings
genotyped with the SolCAP Infinium 8303 potato array. The dataset also
contains the genomic order of the SNPs from the \emph{Solanum tuberosum}
genome version 4.03. The genotype calling was performed with fitPoly R
package using
\href{https://github.com/mmollina/Autopolyploid_Linkage/blob/master/src/solcap_map_construction/snp_calling/genotype_calling_public_data_fittetra.R}{this
pipeline}. Another option would be to use ClusterCall and
\href{https://mmollina.github.io/tutorials/solcap/solcap_example.html}{this
pipeline}.

Once the data is loaded, you can explore the dataset using the
\texttt{print} function:

\begin{Shaded}
\begin{Highlighting}[]
\KeywordTok{print}\NormalTok{(dat.dose.mpl, }\DataTypeTok{detailed =} \OtherTok{TRUE}\NormalTok{)}
\end{Highlighting}
\end{Shaded}

This function outputs information about the dataset including the ploidy
level, total number of individuals, total number of markers, number of
informative markers, proportion of missing data and redundant markers.
If \texttt{detailed\ =\ TRUE}, the function also outputs the number of
markers in each sequence, if available, and the number of markers
contained in all possible dosage combinations between both parents.

You can also explore the dataset visually using the \texttt{plot}
function:

\begin{Shaded}
\begin{Highlighting}[]
\KeywordTok{plot}\NormalTok{(dat.dose.mpl)}
\end{Highlighting}
\end{Shaded}

The output figure shows a bar plot on the left-hand side with the number
of markers contained in each allele dosage combination between both
parents. The right labels indicate allele dosages for Parent 1 and
Parent 2, respectively. The upper-right plot contains the
\(\log_{10}(p-value)\) from \(\chi^2\) tests for all markers,
considering the expected segregation patterns under Mendelian
inheritance. The lower-right plot contains a graphical representation of
the allele dosages and missing data distribution for all markers and
individuals. Finally, the bottom-right graphic shows the proportion of
redundant markers in the dataset, when available.

\hypertarget{marker-specific}{%
\subsubsection{Marker-specific}\label{marker-specific}}

If you want to view a specific marker information, use the
\texttt{plot\_mrk\_info} function. You should indicate your dataset
object using the \texttt{input.data} argument, and the desired marker
using the \texttt{mrk} argument. You can indicate the marker using its
number or its name (string):

\begin{Shaded}
\begin{Highlighting}[]
\CommentTok{# For a dosage-based data analysis of marker 104}
\KeywordTok{plot_mrk_info}\NormalTok{(}\DataTypeTok{input.data =}\NormalTok{ dat.dose.mpl, }\DataTypeTok{mrk =} \DecValTok{240}\NormalTok{)}
\end{Highlighting}
\end{Shaded}

\begin{Shaded}
\begin{Highlighting}[]
\CommentTok{# For a probability-based data analysis of the marker solcap_snp_c1_13686}
\KeywordTok{plot_mrk_info}\NormalTok{(}\DataTypeTok{input.data =}\NormalTok{ dat.prob.mpl, }\DataTypeTok{mrk =} \StringTok{'solcap_snp_c1_13686'}\NormalTok{)}
\end{Highlighting}
\end{Shaded}

When applied to a dosage-based dataset, the function outputs a figure
showing: marker name and position in the dataset, allele dosage in
parents 1 and 2, proportion of missing data, p-value of the associated
\(\chi^2\) test for Mendelian segregation, sequence and position
information (when available). The figure also contains a plot with the
allele dosage and missing data distibution in the population.

When applied to a probability-baed dataset, the function outputs the
probability threshold and a 3-dimensional plot containing the
probability distribution for each allele dosage considering all
individuals.

\hypertarget{filtering-and-quality-control}{%
\section{Filtering and Quality
Control}\label{filtering-and-quality-control}}

In order to build a good genetic map, good quality data is desired to
guarantee reliable estimates of recombination fractions, linkage phases
and haplotypes. High proportions of messy data, such as unexpected
segregation patterns and missing data, will reduce the reliability of
these estimates. Furthermore, sequencing technologies are able to
produce hundreds of thousands of markers, and a considerable proportion
of redundant markers is expected. Markers that carry the same
information are not informative and must be removed from the dataset, in
order to reduce computational effort. \texttt{MAPpoly} handle some
filtering functions that are described in the sections below.

\hypertarget{missing-data-filtering}{%
\subsection{Missing data filtering}\label{missing-data-filtering}}

\texttt{MAPpoly} is able to filter markers and/or individuals that
exceeds a defined threshold for missing data. The function
\texttt{filter\_missing} does that and creates or updates your dataset
according to its arguments' values. The argument \texttt{input.data}
should contain your dataset object, and you can choose to filter either
by `marker' or `individual' using the \texttt{type} argument (string).
You can also define the maximum proportion of missing data using the
\texttt{filter.thres} argument (ranges from 0 to 1, i.e.~a threshold of
0.2 will keep just markers or individuals with less than 20\% of missing
data). When TRUE (default), the \texttt{inter} argument plots markers or
individuals vs.~frequency of missing data.

\begin{Shaded}
\begin{Highlighting}[]
\CommentTok{# Filtering dataset by marker}
\NormalTok{dat.filt.mrk <-}\StringTok{ }\KeywordTok{filter_missing}\NormalTok{(}\DataTypeTok{input.data =}\NormalTok{ dat.dose.mpl, }\DataTypeTok{type =} \StringTok{"marker"}\NormalTok{, }
                               \DataTypeTok{filter.thres =} \FloatTok{0.2}\NormalTok{, }\DataTypeTok{inter =} \OtherTok{TRUE}\NormalTok{)}
\end{Highlighting}
\end{Shaded}

\begin{Shaded}
\begin{Highlighting}[]
\KeywordTok{print}\NormalTok{(dat.filt.mrk)}
\end{Highlighting}
\end{Shaded}

\begin{Shaded}
\begin{Highlighting}[]
\CommentTok{# Filtering dataset by individual}
\NormalTok{dat.filt.ind <-}\StringTok{ }\KeywordTok{filter_missing}\NormalTok{(}\DataTypeTok{input.data =}\NormalTok{ dat.filt.mrk, }\DataTypeTok{type =} \StringTok{"individual"}\NormalTok{, }
                               \DataTypeTok{filter.thres =} \FloatTok{0.1}\NormalTok{, }\DataTypeTok{inter =} \OtherTok{TRUE}\NormalTok{)                            }
\end{Highlighting}
\end{Shaded}

\begin{Shaded}
\begin{Highlighting}[]
\KeywordTok{print}\NormalTok{(dat.filt.ind)}
\end{Highlighting}
\end{Shaded}

\begin{Shaded}
\begin{Highlighting}[]
\NormalTok{dat.dose.filt <-}\StringTok{ }\NormalTok{dat.filt.ind}
\end{Highlighting}
\end{Shaded}

In this dataset, just 43 markers presented a proportion of missing data
above the defined threshold, while 16 individuals exceeded the defined
threshold. Then we will use the final filtered dataset during the rest
of the analysis. Please notice that the function \texttt{read\_vcf} also
provides parameters to filter out markers depending on their average
depth and missing data, as well as removes data points that do not reach
a minimum pre-defined depth value. Check the
\protect\hyperlink{read_vcf}{\texttt{read\_vcf}} section for more
information.

\hypertarget{segregation-test}{%
\subsection{Segregation test}\label{segregation-test}}

Another very important point to consider is the expected marker
segregation pattern under Mendelian inheritance. Markers with messy or
distorted segregation can produce unreliable estimates and may be
removed (at least for a while) from the dataset. A good test for that is
the chi-square (\(\chi^2\)), which basically matches expected genotype
frequencies against observed frequencies and calculates the associated
p-value. In order to define the p-value threshold for the tests, we will
use the Bonferroni correction:

\[\alpha_{thres} = \frac{\alpha}{\#markers}\]

We will also assume that only random chromosome bivalent pairing occurs
and there is no double reduction.

\begin{Shaded}
\begin{Highlighting}[]
\NormalTok{pval.bonf <-}\StringTok{ }\FloatTok{0.05}\OperatorTok{/}\NormalTok{dat.dose.filt}\OperatorTok{$}\NormalTok{n.mrk}
\NormalTok{mrks.chi.filt <-}\StringTok{ }\KeywordTok{filter_segregation}\NormalTok{(dat.dose.filt, }\DataTypeTok{chisq.pval.thres =}\NormalTok{  pval.bonf, }\DataTypeTok{inter =} \OtherTok{TRUE}\NormalTok{)}
\NormalTok{seq.init <-}\StringTok{ }\KeywordTok{make_seq_mappoly}\NormalTok{(mrks.chi.filt)}
\end{Highlighting}
\end{Shaded}

Please notice that \texttt{filter\_segregation} does not produce a
filtered dataset; it just tells you which markers follow the expected
Mendelian segregation pattern. To select these markers from your
dataset, you may use the \texttt{make\_seq\_mappoly} function.

\begin{Shaded}
\begin{Highlighting}[]
\KeywordTok{plot}\NormalTok{(seq.init)}
\end{Highlighting}
\end{Shaded}

It is worth to mention that all redundant markers identified during data
reading step are stored in the main dataset. The redundant markers are
automatically removed and can be added back once the maps are finished
using the function \texttt{update\_map} (described in details later
during this tutorial).

\hypertarget{two-point-analysis}{%
\section{Two-point analysis}\label{two-point-analysis}}

Once the markers are filtered and selected, we need to compute the
pairwise recombination fraction between all of them (two-point
analysis). First, let us load the genotype counts
(\(\zeta_{\mbox{T}_{k},\mbox{T}_{k^{\prime}}}(l_{P}, l_{Q})\)) defined
in equation 20 in \href{https://doi.org/10.1534/g3.119.400378}{Mollinari
and Garcia, 2019}. This object is fundamental to perform the dimension
reduction of the transition space.

\begin{Shaded}
\begin{Highlighting}[]
\NormalTok{counts <-}\StringTok{ }\KeywordTok{cache_counts_twopt}\NormalTok{(}\DataTypeTok{input.seq =}\NormalTok{ seq.init, }\DataTypeTok{cached =} \OtherTok{TRUE}\NormalTok{)}
\end{Highlighting}
\end{Shaded}

\begin{Shaded}
\begin{Highlighting}[]
\NormalTok{counts}
\end{Highlighting}
\end{Shaded}

When \texttt{cached\ =\ TRUE}, the function loads all counts from a
internal file instead of performing all calculations again.

Next, the function \texttt{est\_pairwise\_rf} is used to estimate all
the pairwise recombination fractions between markers in the sequence
provided. Since the output object is too big to be fully displayed on
the screen, \texttt{MAPpoly} shows a summary. Please notice that
parallel computation is available and, if you want to use of that, you
need to define the available number of cores in your machine and also
guarantee that you have sufficient RAM memory for that. Remember that it
is always good to leave one core available for the system, and be aware
that this step \textbf{will take a while} to compute if you have few
cores available.

\begin{Shaded}
\begin{Highlighting}[]
\CommentTok{# Defining number of cores}
\NormalTok{n.cores =}\StringTok{ }\NormalTok{parallel}\OperatorTok{::}\KeywordTok{detectCores}\NormalTok{() }\OperatorTok{-}\StringTok{ }\DecValTok{1}

\CommentTok{#(~ 9.5 minutes using 14 cores)}
\NormalTok{all.rf.pairwise <-}\StringTok{ }\KeywordTok{est_pairwise_rf}\NormalTok{(}\DataTypeTok{input.seq =}\NormalTok{ seq.init, }
                                   \DataTypeTok{count.cache =}\NormalTok{ counts, }
                                   \DataTypeTok{n.clusters =}\NormalTok{ n.cores)}
\end{Highlighting}
\end{Shaded}

\begin{Shaded}
\begin{Highlighting}[]
\NormalTok{all.rf.pairwise}
\end{Highlighting}
\end{Shaded}

To assess the recombination fraction between a particular pair of
markers, say markers 93 and 98, we use the following syntax:

\begin{Shaded}
\begin{Highlighting}[]
\NormalTok{all.rf.pairwise}\OperatorTok{$}\NormalTok{pairwise}\OperatorTok{$}\StringTok{`}\DataTypeTok{93-98}\StringTok{`}
\KeywordTok{plot}\NormalTok{(all.rf.pairwise, }\DataTypeTok{first.mrk =} \DecValTok{93}\NormalTok{, }\DataTypeTok{second.mrk =} \DecValTok{98}\NormalTok{)}
\end{Highlighting}
\end{Shaded}

In this case, \texttt{93-98} represents the position of the markers in
the filtered data set. The name of the rows have the form \texttt{x-y},
where \texttt{x} and \texttt{y} indicate how many homologous chromosomes
share the same allelic variant in parents \(P1\) and \(P2\),
respectively (see \href{https://doi.org/10.1534/g3.119.400378}{Mollinari
and Garcia, 2019} for notation). The first column indicates the LOD
Score in relation to the most likely linkage phase configuration. The
second column shows the estimated recombination fraction for each
configuration, and the third indicates the LOD Score comparing the
likelihood under no linkage (\(r = 0.5\)) with the estimated
recombination fraction (evidence of linkage).

\hypertarget{assembling-recombination-fraction-and-lod-score-matrices}{%
\subsection{Assembling recombination fraction and LOD Score
matrices}\label{assembling-recombination-fraction-and-lod-score-matrices}}

Recombination fraction and LOD Score matrices are fundamental in genetic
mapping. Later in this tutorial, we will use these matrices as the basic
information to order markers and also to perform some diagnostics. To
convert the two-point object into recombination fraction and LOD Score
matrices, we need to assume thresholds for the three columns observed in
the previous output. The arguments \texttt{thresh.LOD.ph} and
\texttt{thresh.LOD.rf} set LOD Scores thresholds for the second most
likely linkage phase configuration and recombination fraction. Here we
assume \texttt{thresh.LOD.ph\ =\ 0} and \texttt{thresh.LOD.rf\ =\ 0},
thus no matter how likely is the second best option, all the computed
values will be considered. The argument \texttt{thresh.rf\ =\ 0.5}
indicates that the maximum accepted recombination fraction is
\texttt{0.5}. To convert these values in a recombination fraction
matrix, we use the function \texttt{rf\_list\_to\_matrix}.

\begin{Shaded}
\begin{Highlighting}[]
\NormalTok{mat <-}\StringTok{ }\KeywordTok{rf_list_to_matrix}\NormalTok{(}\DataTypeTok{input.twopt =}\NormalTok{ all.rf.pairwise, }\DataTypeTok{n.clusters =}\NormalTok{ n.cores)}
\end{Highlighting}
\end{Shaded}

Please notice that for big datasets, you may use the multi-core support
to perform the conversions, using the parameter \texttt{n.clusters} to
define the number of CPU's. It is also possible to filter again for the
parameters mentioned above, such as \texttt{thresh.LOD.ph},
\texttt{thresh.LOD.rf} and \texttt{thresh.rf}.

We can also plot this matrix using the reference genome order. For doing
so, we use the function \texttt{get\_genomic\_order} to get the genomic
order of the input sequence and use the resulting order to index the
recombination fraction matrix. If the reference order is consistent with
the marker order in this specific population, we should observe a
block-diagonal matrix and within each sub-matrix, a monotonic pattern.

\begin{Shaded}
\begin{Highlighting}[]
\NormalTok{id<-}\KeywordTok{get_genomic_order}\NormalTok{(seq.init)}
\KeywordTok{plot}\NormalTok{(mat, }\DataTypeTok{ord =} \KeywordTok{rownames}\NormalTok{(id), }\DataTypeTok{index =} \OtherTok{FALSE}\NormalTok{)}
\end{Highlighting}
\end{Shaded}

As expected, we can observe the block-diagonal and monotonic patterns.
In the previous case, the thresholds allowed to plot almost all points
in the recombination fraction matrix. The empty cells in the matrix
indicate markers where it is impossible to detect recombinant events
using two-point estimates (e.g., between \(1 \times 0\) and
\(0 \times 1\) marker). Yet, if the thresholds become more stringent
(higher LODs and lower rf), the matrix becomes more sparse.

\hypertarget{assembling-linkage-groups}{%
\section{Assembling linkage groups}\label{assembling-linkage-groups}}

The function \texttt{group\_mappoly} assign markers to linkage groups
using the recombination fraction matrix obtained above. The user can
provide an expected number of groups or run the interactive version of
the function using \texttt{inter\ =\ TRUE}. Since in this data set we
expect 12 linkage groups (basic chromosome number in potato), we use
\texttt{expected.groups\ =\ 12}. If the data set provides the chromosome
where the markers are located, the function allows comparing the groups
obtained using the pairwise recombination fraction and the chromosome
information provided using the \texttt{comp.mat\ =\ TRUE}.

\begin{Shaded}
\begin{Highlighting}[]
\NormalTok{grs <-}\StringTok{ }\KeywordTok{group_mappoly}\NormalTok{(}\DataTypeTok{input.mat =}\NormalTok{ mat,}
                     \DataTypeTok{expected.groups =} \DecValTok{12}\NormalTok{,}
                     \DataTypeTok{comp.mat =} \OtherTok{TRUE}\NormalTok{, }
                     \DataTypeTok{inter =} \OtherTok{TRUE}\NormalTok{)}
\end{Highlighting}
\end{Shaded}

\begin{Shaded}
\begin{Highlighting}[]
\NormalTok{grs}
\end{Highlighting}
\end{Shaded}

Here, we have the 3639 markers distributed in 12 linkage groups. The
rows indicate linkage groups obtained using linkage information and the
columns are the chromosomes in the reference genome. Notice the diagonal
indicating the concordance between the two sources of information. Now,
we can plot the resulting marker cluster analysis.

\begin{Shaded}
\begin{Highlighting}[]
\KeywordTok{plot}\NormalTok{(grs)}
\end{Highlighting}
\end{Shaded}

Once the linkage groups are properly assembled, we use the function
\texttt{make\_seq\_mappoly} to make marker sequences from the group
analysis. We will assemble a list with 12 positions, each one containing
the corresponding linkage group sequence. Also, we will use only markers
allocated in the diagonal of the previous comparison matrix. Thus only
markers that were assigned to a particular linkage group using both
sources of information will be considered. We will also assemble a
smaller two-point object using the functions
\texttt{make\_pairs\_mappoly} and \texttt{rf\_snp\_filter} to facilitate
further parallelization procedures.

\begin{Shaded}
\begin{Highlighting}[]
\NormalTok{LGS<-}\KeywordTok{vector}\NormalTok{(}\StringTok{"list"}\NormalTok{, }\DecValTok{12}\NormalTok{)}
\ControlFlowTok{for}\NormalTok{(j }\ControlFlowTok{in} \DecValTok{1}\OperatorTok{:}\DecValTok{12}\NormalTok{)\{}
\NormalTok{    temp1 <-}\StringTok{ }\KeywordTok{make_seq_mappoly}\NormalTok{(grs, j)}
\NormalTok{    temp2 <-}\StringTok{ }\KeywordTok{get_genomic_order}\NormalTok{(temp1) }\CommentTok{# assembling sequence considering the genomic order}
\NormalTok{    lg.id <-}\StringTok{ }\KeywordTok{as.numeric}\NormalTok{(}\KeywordTok{names}\NormalTok{(}\KeywordTok{which.max}\NormalTok{(}\KeywordTok{table}\NormalTok{(temp2[,}\DecValTok{1}\NormalTok{]))))}
\NormalTok{    nm <-}\StringTok{ }\KeywordTok{rownames}\NormalTok{(temp2)[}\KeywordTok{which}\NormalTok{(temp2[,}\DecValTok{1}\NormalTok{] }\OperatorTok{==}\StringTok{ }\NormalTok{lg.id)]}
\NormalTok{    temp3 <-}\StringTok{ }\KeywordTok{make_seq_mappoly}\NormalTok{(dat.dose.filt, nm)}
\NormalTok{    tpt <-}\StringTok{ }\KeywordTok{make_pairs_mappoly}\NormalTok{(all.rf.pairwise, }\DataTypeTok{input.seq =}\NormalTok{ temp3)}
\NormalTok{    lgtemp <-}\StringTok{ }\KeywordTok{rf_snp_filter}\NormalTok{(}\DataTypeTok{input.twopt =}\NormalTok{ tpt)}
\NormalTok{    LGS[[lg.id]] <-}\StringTok{ }\KeywordTok{list}\NormalTok{(}\DataTypeTok{lg =}\NormalTok{ lgtemp, }
        \DataTypeTok{tpt =} \KeywordTok{make_pairs_mappoly}\NormalTok{(all.rf.pairwise, }\DataTypeTok{input.seq =}\NormalTok{ lgtemp))}
\NormalTok{\}}
\end{Highlighting}
\end{Shaded}

Now, let us print the recombination fraction matrices for each linkage
group.

\hypertarget{estimating-the-map-for-a-given-order}{%
\section{Estimating the map for a given
order}\label{estimating-the-map-for-a-given-order}}

In this section, we will use the marker order provided by the
\emph{Solanum tuberosum} genome version 4.03. The MDS ordering approach
will be addressed later in this tutorial. The estimation of the genetic
map for a given order involves the computation of recombination fraction
between adjacent markers and also finding the linkage phase
configuration of those markers in both parents. The core function to
perform these tasks in \texttt{MAPpoly} is
\texttt{est\_rf\_hmm\_sequential}. This function uses the pairwise
recombination fraction as the first source of information to
sequentially position allelic variants in specific homologs. For
situations where pairwise analysis has limited power, the algorithm
relies on the likelihood obtained through a hidden Markov model (HMM)
\href{https://doi.org/10.1534/g3.119.400378}{Mollinari and Garcia,
2019}. Once all markers are positioned, the final map is reconstructed
using the HMM multipoint algorithm.

Several arguments are available to control the inclusion and phasing of
the markers in the chain. The argument \texttt{start.set} defines the
number of initial markers to be used in a exhaustive search for the most
probable configuration. \texttt{thres.twopt} receives the threshold to
whether when the linkage phases compared via two-point analysis should
be considered, and the HMM analysis should not be used to infer the
linkage phase (A. K. A. \(\eta\) in
\href{https://doi.org/10.1534/g3.119.400378}{Mollinari and Garcia,
2019}). \texttt{thres.hmm} receives the threshold for keeping competing
maps computed using HMM (if the two-point analysis was not enough) in
the next round of marker insertion. \texttt{extend.tail} indicates the
number of markers that should be considered at the end of the chain to
insert a new marker. \texttt{tol} and \texttt{tol.final} receive the
desired accuracy to estimate the sub-maps during the sequential phasing
procedure and the desired accuracy in the final map.
\texttt{phase.number.limit} receives the limit number of linkage phase
configurations to be tested using HMM. \texttt{info.tail} is a logical
argument: if TRUE it uses the complete informative tail (last markers in
the chain that allow all homologous to be distinguished in the parents)
of the chain to calculate the likelihood of the linkage phases.

First, as an example, let us estimate the map for linkage group 3. The
values used for all arguments were obtained using a balance of
processing speed and accuracy of the algorithm. As an exercise, it is
interesting to try different values and check out the results. For now,
let us stick with the following values (\textbf{this step can take some
hours to finish}, depending on chosen parameters):

\begin{Shaded}
\begin{Highlighting}[]
\NormalTok{lg3.map <-}\StringTok{ }\KeywordTok{est_rf_hmm_sequential}\NormalTok{(}\DataTypeTok{input.seq =}\NormalTok{ LGS[[}\DecValTok{3}\NormalTok{]]}\OperatorTok{$}\NormalTok{lg,}
                                \DataTypeTok{start.set =} \DecValTok{10}\NormalTok{,}
                                \DataTypeTok{thres.twopt =} \DecValTok{10}\NormalTok{, }
                                \DataTypeTok{thres.hmm =} \DecValTok{10}\NormalTok{,}
                                \DataTypeTok{extend.tail =} \DecValTok{200}\NormalTok{,}
                                \DataTypeTok{info.tail =} \OtherTok{TRUE}\NormalTok{, }
                                \DataTypeTok{twopt =}\NormalTok{ LGS[[}\DecValTok{3}\NormalTok{]]}\OperatorTok{$}\NormalTok{tpt,}
                                \DataTypeTok{sub.map.size.diff.limit =} \DecValTok{10}\NormalTok{, }
                                \DataTypeTok{phase.number.limit =} \DecValTok{20}\NormalTok{,}
                                \DataTypeTok{reestimate.single.ph.configuration =} \OtherTok{TRUE}\NormalTok{,}
                                \DataTypeTok{tol =} \FloatTok{10e-3}\NormalTok{,}
                                \DataTypeTok{tol.final =} \FloatTok{10e-4}\NormalTok{)}
\end{Highlighting}
\end{Shaded}

Now, use the functions \texttt{print} and \texttt{plot} to view the
results:

\begin{Shaded}
\begin{Highlighting}[]
\KeywordTok{print}\NormalTok{(lg3.map)}
\KeywordTok{plot}\NormalTok{(lg3.map)}
\end{Highlighting}
\end{Shaded}

Colored rectangles (red and blue) indicate the presence of the allelic
variant in each one of the four homologous in both parents, \(P_1\) and
\(P_2\).

\hypertarget{reestimating-the-map-considering-genotyping-errors}{%
\subsection{Reestimating the map considering genotyping
errors}\label{reestimating-the-map-considering-genotyping-errors}}

Though current technologies enabled the genotyping of thousands of SNPs,
they are quite prone to genotyping errors, especially in polyploid
species. One way to address this problem is to associate a probability
distribution to each one of the markers and allow the HMM to update
their probability. This procedure can be applied using either the
probability distribution provided by the genotype calling software
(loaded in \texttt{MAPpoly} using the function
\texttt{read\_geno\_prob}) or assuming a global genotype error. For a
detailed explanation of this procedure, please see
\href{https://doi.org/10.1534/g3.119.400378}{Mollinari and Garcia,
2019}. Briefly, the use of a prior information will update the genotype
of the markers based on a global chromosome structure. In this tutorial,
since we are analyzing the dosage data with no probability distribution
associated, the second approch will be used.

\begin{Shaded}
\begin{Highlighting}[]
\NormalTok{lg3.map.error <-}\StringTok{ }\KeywordTok{est_full_hmm_with_global_error}\NormalTok{(}\DataTypeTok{input.map =}\NormalTok{ lg3.map, }\DataTypeTok{error =} \FloatTok{0.05}\NormalTok{)}
\end{Highlighting}
\end{Shaded}

\begin{Shaded}
\begin{Highlighting}[]
\KeywordTok{plot}\NormalTok{(lg3.map.error)}
\end{Highlighting}
\end{Shaded}

Notice that a global genotyping error of 5\% was used, and the resulting
map was smaller than the previous one. Also, some markers were
``attracted'' and some markers were ``repealed'' as a result of the
smaller confidence used for each marker genotype.

\hypertarget{reinserting-redundant-markers}{%
\subsection{Reinserting redundant
markers}\label{reinserting-redundant-markers}}

As mentioned before, redundant markers are automatically removed during
the analysis to reduce computational calculations. Besides that, one may
want to see the genetic linkage maps considering all markers in the full
dataset, including the redundant ones. The addition of redundant markers
to a map can be done by just calling the function \texttt{update\_map}.
Here we will show the addition in the previous map:

\begin{Shaded}
\begin{Highlighting}[]
\NormalTok{lg3.map.updated =}\StringTok{ }\KeywordTok{update_map}\NormalTok{(lg3.map)}
\NormalTok{lg3.map}
\NormalTok{lg3.map.updated}
\end{Highlighting}
\end{Shaded}

As can be seen, both maps are identical except for the number of
markers.

\hypertarget{ordering-markers-using-mds-and-reestimating-the-map}{%
\section{Ordering markers using MDS and reestimating the
map}\label{ordering-markers-using-mds-and-reestimating-the-map}}

So far the map was reestimated using the genomic order. In real
situations, unless a genomic information is provided, the markers need
to be ordered using an optimization technique. Here, we use the MDS
(multidimensional scaling) algorithm, proposed in the context of genetic
mapping by Preedy and Hackett (2016). The MDS algorithm requires a
recombination fraction matrix, which will be transformed in distances
using a mapping function (in this case the Haldane's mapping function).
First, let us gather the pairwise recombination fractions for all three
linkage groups:

\begin{Shaded}
\begin{Highlighting}[]
\NormalTok{mt <-}\StringTok{ }\KeywordTok{lapply}\NormalTok{(LGS, }\ControlFlowTok{function}\NormalTok{(x) }\KeywordTok{rf_list_to_matrix}\NormalTok{(x}\OperatorTok{$}\NormalTok{tpt))}
\end{Highlighting}
\end{Shaded}

Now, for each matrix contained in the object \texttt{mt}, we use the MDS
algorithm:

\begin{Shaded}
\begin{Highlighting}[]
\NormalTok{mds.ord <-}\StringTok{ }\KeywordTok{lapply}\NormalTok{(mt, mds_mappoly)}
\end{Highlighting}
\end{Shaded}

Usually at this point, the user can make use of diagnostic plots to
remove markers that are disturbing the ordering procedure. Here we
didn't use that procedure, but we encourage the user to check the
example in \texttt{?mds\_mappoly}. Now, let us compare the estimated and
the genomic orders (feel free to run the last commented line and see
interactive plots):

\begin{Shaded}
\begin{Highlighting}[]
\NormalTok{LGS.mds<-}\KeywordTok{vector}\NormalTok{(}\StringTok{"list"}\NormalTok{, }\DecValTok{12}\NormalTok{)}
\ControlFlowTok{for}\NormalTok{(j }\ControlFlowTok{in} \DecValTok{1}\OperatorTok{:}\DecValTok{12}\NormalTok{)\{}
\NormalTok{  lgtemp <-}\StringTok{ }\KeywordTok{make_seq_mappoly}\NormalTok{(mds.ord[[j]])}
\NormalTok{  LGS.mds[[j]] <-}\StringTok{ }\KeywordTok{list}\NormalTok{(}\DataTypeTok{lg =}\NormalTok{ lgtemp, }
      \DataTypeTok{tpt =} \KeywordTok{make_pairs_mappoly}\NormalTok{(all.rf.pairwise, }\DataTypeTok{input.seq =}\NormalTok{ lgtemp))}
\NormalTok{\}}
\end{Highlighting}
\end{Shaded}

\begin{Shaded}
\begin{Highlighting}[]
\NormalTok{geno.vs.mds <-}\StringTok{ }\OtherTok{NULL}
\ControlFlowTok{for}\NormalTok{(i }\ControlFlowTok{in} \DecValTok{1}\OperatorTok{:}\KeywordTok{length}\NormalTok{(LGS.mds))\{}
\NormalTok{  geno.vs.mds<-}\KeywordTok{rbind}\NormalTok{(geno.vs.mds,}
                     \KeywordTok{data.frame}\NormalTok{(}\DataTypeTok{mrk.names =}\NormalTok{ LGS.mds[[i]]}\OperatorTok{$}\NormalTok{lg}\OperatorTok{$}\NormalTok{seq.mrk.names,}
                                \DataTypeTok{mds.pos =} \KeywordTok{seq_along}\NormalTok{(LGS.mds[[i]]}\OperatorTok{$}\NormalTok{lg}\OperatorTok{$}\NormalTok{seq.mrk.names),}
                                \DataTypeTok{genomic.pos =} \KeywordTok{order}\NormalTok{(LGS.mds[[i]]}\OperatorTok{$}\NormalTok{lg}\OperatorTok{$}\NormalTok{sequence.pos),}
                                \DataTypeTok{LG =} \KeywordTok{paste0}\NormalTok{(}\StringTok{"LG_"}\NormalTok{, i)))}
\NormalTok{\}}

\KeywordTok{require}\NormalTok{(ggplot2)}
\NormalTok{p<-}\KeywordTok{ggplot}\NormalTok{(geno.vs.mds, }\KeywordTok{aes}\NormalTok{(genomic.pos, mds.pos)) }\OperatorTok{+}
\StringTok{  }\KeywordTok{geom_point}\NormalTok{(}\DataTypeTok{alpha =} \DecValTok{1}\OperatorTok{/}\DecValTok{5}\NormalTok{, }\KeywordTok{aes}\NormalTok{(}\DataTypeTok{colour =}\NormalTok{ LG)) }\OperatorTok{+}
\StringTok{  }\KeywordTok{facet_wrap}\NormalTok{(}\OperatorTok{~}\NormalTok{LG) }\OperatorTok{+}\StringTok{  }\KeywordTok{xlab}\NormalTok{(}\StringTok{"Genome Order"}\NormalTok{) }\OperatorTok{+}\StringTok{ }\KeywordTok{ylab}\NormalTok{(}\StringTok{"Map Order"}\NormalTok{)}
\NormalTok{p}
\CommentTok{#plotly::ggplotly(p)}
\end{Highlighting}
\end{Shaded}

Although several local inconsistencies occurred, the global diagonal
pattern indicates a consistent order for all linkage groups using both
approaches. Now, let us build the genetic map of linkage group 3 using
the MDS order (remember that \textbf{this can take a while to finish}):

\begin{Shaded}
\begin{Highlighting}[]
\NormalTok{lg3.map.mds <-}\StringTok{ }\KeywordTok{est_rf_hmm_sequential}\NormalTok{(}\DataTypeTok{input.seq =}\NormalTok{ LGS.mds[[}\DecValTok{3}\NormalTok{]]}\OperatorTok{$}\NormalTok{lg,}
                                \DataTypeTok{start.set =} \DecValTok{10}\NormalTok{,}
                                \DataTypeTok{thres.twopt =} \DecValTok{10}\NormalTok{, }
                                \DataTypeTok{thres.hmm =} \DecValTok{10}\NormalTok{,}
                                \DataTypeTok{extend.tail =} \DecValTok{200}\NormalTok{,}
                                \DataTypeTok{info.tail =} \OtherTok{TRUE}\NormalTok{, }
                                \DataTypeTok{twopt =}\NormalTok{ LGS.mds[[}\DecValTok{3}\NormalTok{]]}\OperatorTok{$}\NormalTok{tpt,}
                                \DataTypeTok{sub.map.size.diff.limit =} \DecValTok{10}\NormalTok{, }
                                \DataTypeTok{phase.number.limit =} \DecValTok{20}\NormalTok{,}
                                \DataTypeTok{reestimate.single.ph.configuration =} \OtherTok{TRUE}\NormalTok{,}
                                \DataTypeTok{tol =} \FloatTok{10e-3}\NormalTok{,}
                                \DataTypeTok{tol.final =} \FloatTok{10e-4}\NormalTok{)}
\end{Highlighting}
\end{Shaded}

And plot the map:

\begin{Shaded}
\begin{Highlighting}[]
\KeywordTok{plot}\NormalTok{(lg3.map.mds)}
\end{Highlighting}
\end{Shaded}

It is also possible to compare the maps using both genomic-based and
MDS-based orders with the function \texttt{plot\_map\_list}:

\begin{Shaded}
\begin{Highlighting}[]
\KeywordTok{plot_map_list}\NormalTok{(}\KeywordTok{list}\NormalTok{(lg3.map, lg3.map.mds), }\DataTypeTok{col =} \KeywordTok{c}\NormalTok{(}\StringTok{"red"}\NormalTok{, }\StringTok{"blue"}\NormalTok{), }\DataTypeTok{title =} \StringTok{""}\NormalTok{)}
\end{Highlighting}
\end{Shaded}

The genomic-based map included the same number of markers but is smaller
than the MDS-based map, which indicates a better result. To formally
compare the maps, one needs to select the markers that are present in
both maps. Interestingly enough, both procedures included the same
markers in the final map; however, we provide the code to perform the
comparison even if the maps share only a subset of markers:

\begin{Shaded}
\begin{Highlighting}[]
\NormalTok{mrks.in.gen<-}\KeywordTok{intersect}\NormalTok{(lg3.map}\OperatorTok{$}\NormalTok{maps[[}\DecValTok{1}\NormalTok{]]}\OperatorTok{$}\NormalTok{seq.num, lg3.map.mds}\OperatorTok{$}\NormalTok{maps[[}\DecValTok{1}\NormalTok{]]}\OperatorTok{$}\NormalTok{seq.num)}
\NormalTok{mrks.in.mds<-}\KeywordTok{intersect}\NormalTok{(lg3.map.mds}\OperatorTok{$}\NormalTok{maps[[}\DecValTok{1}\NormalTok{]]}\OperatorTok{$}\NormalTok{seq.num, lg3.map}\OperatorTok{$}\NormalTok{maps[[}\DecValTok{1}\NormalTok{]]}\OperatorTok{$}\NormalTok{seq.num)}
\ControlFlowTok{if}\NormalTok{(}\KeywordTok{cor}\NormalTok{(mrks.in.gen, mrks.in.mds) }\OperatorTok{<}\StringTok{ }\DecValTok{0}\NormalTok{)\{}
\NormalTok{  mrks.in.mds <-}\StringTok{ }\KeywordTok{rev}\NormalTok{(mrks.in.mds)}
\NormalTok{  lg3.map.mds <-}\StringTok{ }\KeywordTok{rev_map}\NormalTok{(lg3.map.mds)}
\NormalTok{\}}
\NormalTok{map.comp.}\FloatTok{3.}\NormalTok{gen<-}\KeywordTok{get_submap}\NormalTok{(}\DataTypeTok{input.map =}\NormalTok{ lg3.map, }\KeywordTok{match}\NormalTok{(mrks.in.gen, lg3.map}\OperatorTok{$}\NormalTok{maps[[}\DecValTok{1}\NormalTok{]]}\OperatorTok{$}\NormalTok{seq.num), }\DataTypeTok{verbose =} \OtherTok{FALSE}\NormalTok{)}
\NormalTok{map.comp.}\FloatTok{3.}\NormalTok{mds<-}\KeywordTok{get_submap}\NormalTok{(}\DataTypeTok{input.map =}\NormalTok{ lg3.map.mds, }\KeywordTok{match}\NormalTok{(mrks.in.mds, lg3.map.mds}\OperatorTok{$}\NormalTok{maps[[}\DecValTok{1}\NormalTok{]]}\OperatorTok{$}\NormalTok{seq.num), }\DataTypeTok{verbose =} \OtherTok{FALSE}\NormalTok{)}
\NormalTok{prob.}\FloatTok{3.}\NormalTok{gen<-}\KeywordTok{extract_map}\NormalTok{(lg3.map)}
\NormalTok{prob.}\FloatTok{3.}\NormalTok{mds<-}\KeywordTok{extract_map}\NormalTok{(lg3.map.mds)}
\KeywordTok{names}\NormalTok{(prob.}\FloatTok{3.}\NormalTok{gen)<-map.comp.}\FloatTok{3.}\NormalTok{gen}\OperatorTok{$}\NormalTok{maps[[}\DecValTok{1}\NormalTok{]]}\OperatorTok{$}\NormalTok{seq.num}
\KeywordTok{names}\NormalTok{(prob.}\FloatTok{3.}\NormalTok{mds)<-map.comp.}\FloatTok{3.}\NormalTok{mds}\OperatorTok{$}\NormalTok{maps[[}\DecValTok{1}\NormalTok{]]}\OperatorTok{$}\NormalTok{seq.num}
\end{Highlighting}
\end{Shaded}

\begin{Shaded}
\begin{Highlighting}[]
\KeywordTok{matplot}\NormalTok{(}\KeywordTok{t}\NormalTok{(}\KeywordTok{data.frame}\NormalTok{(prob.}\FloatTok{3.}\NormalTok{gen,prob.}\FloatTok{3.}\NormalTok{mds[}\KeywordTok{names}\NormalTok{(prob.}\FloatTok{3.}\NormalTok{gen)])), }
        \DataTypeTok{type=}\StringTok{"b"}\NormalTok{, }\DataTypeTok{pch=}\StringTok{"_"}\NormalTok{, }\DataTypeTok{col=}\DecValTok{1}\NormalTok{, }\DataTypeTok{lty=}\DecValTok{1}\NormalTok{, }\DataTypeTok{lwd =} \FloatTok{.5}\NormalTok{, }\DataTypeTok{xlab=} \StringTok{""}\NormalTok{, }
        \DataTypeTok{ylab=}\StringTok{"Marker position (cM)"}\NormalTok{, }\DataTypeTok{axes =}\NormalTok{ F)}
\KeywordTok{axis}\NormalTok{(}\DecValTok{2}\NormalTok{)}
\KeywordTok{mtext}\NormalTok{(}\DataTypeTok{text =} \KeywordTok{round}\NormalTok{(map.comp.}\FloatTok{3.}\NormalTok{gen}\OperatorTok{$}\NormalTok{maps[[}\DecValTok{1}\NormalTok{]]}\OperatorTok{$}\NormalTok{loglike,}\DecValTok{1}\NormalTok{), }\DataTypeTok{side =} \DecValTok{1}\NormalTok{, }\DataTypeTok{adj =} \DecValTok{0}\NormalTok{)}
\KeywordTok{mtext}\NormalTok{(}\DataTypeTok{text =} \KeywordTok{round}\NormalTok{(map.comp.}\FloatTok{3.}\NormalTok{mds}\OperatorTok{$}\NormalTok{maps[[}\DecValTok{1}\NormalTok{]]}\OperatorTok{$}\NormalTok{loglike,}\DecValTok{1}\NormalTok{), }\DataTypeTok{side =} \DecValTok{1}\NormalTok{, }\DataTypeTok{adj =} \DecValTok{1}\NormalTok{)}
\KeywordTok{mtext}\NormalTok{(}\DataTypeTok{text =} \StringTok{"Genomic"}\NormalTok{, }\DataTypeTok{side =} \DecValTok{3}\NormalTok{, }\DataTypeTok{adj =} \DecValTok{0}\NormalTok{)}
\KeywordTok{mtext}\NormalTok{(}\DataTypeTok{text =} \StringTok{"MDS"}\NormalTok{, }\DataTypeTok{side =} \DecValTok{3}\NormalTok{, }\DataTypeTok{adj =} \DecValTok{1}\NormalTok{)}
\end{Highlighting}
\end{Shaded}

Please notice that these maps have the same local inversions shown in
the dot plots presented earlier. In this case, the log-likelihood of the
genomic order is higher than the one obtained using the MDS order. For
this linkage group, the genomic-based map was chosen as the best one.

\hypertarget{parallel-map-construction}{%
\section{Parallel map construction}\label{parallel-map-construction}}

\hypertarget{using-one-core-by-lg}{%
\subsection{Using one core by LG}\label{using-one-core-by-lg}}

Now, the mapping procedure will be applied to all linkage groups using
parallelization. Although users must be encouraged to compare both MDS
and genomic orders following the previous example, the genomic order
will be considered as an example (remember that this step will take a
\textbf{long time} to run):

\begin{Shaded}
\begin{Highlighting}[]
\CommentTok{## Performing parallel computation}
\NormalTok{my.phase.func<-}\ControlFlowTok{function}\NormalTok{(X)\{}
\NormalTok{  x<-}\KeywordTok{est_rf_hmm_sequential}\NormalTok{(}\DataTypeTok{input.seq =}\NormalTok{ X}\OperatorTok{$}\NormalTok{lg,}
                                \DataTypeTok{start.set =} \DecValTok{10}\NormalTok{,}
                                \DataTypeTok{thres.twopt =} \DecValTok{10}\NormalTok{, }
                                \DataTypeTok{thres.hmm =} \DecValTok{10}\NormalTok{,}
                                \DataTypeTok{extend.tail =} \DecValTok{200}\NormalTok{,}
                                \DataTypeTok{info.tail =} \OtherTok{TRUE}\NormalTok{, }
                                \DataTypeTok{twopt =}\NormalTok{ X}\OperatorTok{$}\NormalTok{tpt,}
                                \DataTypeTok{sub.map.size.diff.limit =} \DecValTok{8}\NormalTok{, }
                                \DataTypeTok{phase.number.limit =} \DecValTok{10}\NormalTok{,}
                                \DataTypeTok{reestimate.single.ph.configuration =} \OtherTok{TRUE}\NormalTok{,}
                                \DataTypeTok{tol =} \FloatTok{10e-3}\NormalTok{,}
                                \DataTypeTok{tol.final =} \FloatTok{10e-4}\NormalTok{)}
  \KeywordTok{return}\NormalTok{(x)}
\NormalTok{\}}
\KeywordTok{system.time}\NormalTok{(\{}
\NormalTok{  cl <-}\StringTok{ }\NormalTok{parallel}\OperatorTok{::}\KeywordTok{makeCluster}\NormalTok{(n.cores)}
\NormalTok{  parallel}\OperatorTok{::}\KeywordTok{clusterEvalQ}\NormalTok{(cl, }\KeywordTok{require}\NormalTok{(mappoly))}
\NormalTok{  parallel}\OperatorTok{::}\KeywordTok{clusterExport}\NormalTok{(cl, }\StringTok{"dat.dose.filt"}\NormalTok{)}
\NormalTok{  MAPs <-}\StringTok{ }\NormalTok{parallel}\OperatorTok{::}\KeywordTok{parLapply}\NormalTok{(cl,LGS,my.phase.func)}
\NormalTok{  parallel}\OperatorTok{::}\KeywordTok{stopCluster}\NormalTok{(cl)}
\NormalTok{\})}
\end{Highlighting}
\end{Shaded}

A traditional linkage map plot can be generated including all linkage
groups, using the function \texttt{plot\_map\_list}:

\begin{Shaded}
\begin{Highlighting}[]
\KeywordTok{plot_map_list}\NormalTok{(MAPs, }\DataTypeTok{col =} \StringTok{"ggstyle"}\NormalTok{)}
\end{Highlighting}
\end{Shaded}

Following the reconstruction of LG 3 shown before, let us consider a
global genotyping error of 5\% to reestimate the final maps:

\begin{Shaded}
\begin{Highlighting}[]
\NormalTok{my.error.func<-}\ControlFlowTok{function}\NormalTok{(X)\{}
\NormalTok{  x<-}\KeywordTok{est_full_hmm_with_global_error}\NormalTok{(}\DataTypeTok{input.map =}\NormalTok{ X, }
                                    \DataTypeTok{error =} \FloatTok{0.05}\NormalTok{, }
                                    \DataTypeTok{tol =} \FloatTok{10e-4}\NormalTok{, }
                                    \DataTypeTok{verbose =} \OtherTok{FALSE}\NormalTok{)}
  \KeywordTok{return}\NormalTok{(x)}
\NormalTok{\}}
\KeywordTok{system.time}\NormalTok{(\{}
\NormalTok{  cl <-}\StringTok{ }\NormalTok{parallel}\OperatorTok{::}\KeywordTok{makeCluster}\NormalTok{(n.cores)}
\NormalTok{  parallel}\OperatorTok{::}\KeywordTok{clusterEvalQ}\NormalTok{(cl, }\KeywordTok{require}\NormalTok{(mappoly))}
\NormalTok{  parallel}\OperatorTok{::}\KeywordTok{clusterExport}\NormalTok{(cl, }\StringTok{"dat.dose.filt"}\NormalTok{)}
\NormalTok{  MAP.err <-}\StringTok{ }\NormalTok{parallel}\OperatorTok{::}\KeywordTok{parLapply}\NormalTok{(cl,MAPs,my.error.func)}
\NormalTok{  parallel}\OperatorTok{::}\KeywordTok{stopCluster}\NormalTok{(cl)}
\NormalTok{\})}
\end{Highlighting}
\end{Shaded}

Comparing both results:

\begin{Shaded}
\begin{Highlighting}[]
\NormalTok{all.MAPs <-}\StringTok{ }\OtherTok{NULL}
\ControlFlowTok{for}\NormalTok{(i }\ControlFlowTok{in} \DecValTok{1}\OperatorTok{:}\DecValTok{12}\NormalTok{) }
\NormalTok{  all.MAPs<-}\KeywordTok{c}\NormalTok{(all.MAPs, MAPs[i], MAP.err[i])}
\KeywordTok{plot_map_list}\NormalTok{(}\DataTypeTok{map.list =}\NormalTok{ all.MAPs, }\DataTypeTok{col =} \KeywordTok{rep}\NormalTok{(}\KeywordTok{c}\NormalTok{(}\StringTok{"#E69F00"}\NormalTok{, }\StringTok{"#56B4E9"}\NormalTok{), }\DecValTok{12}\NormalTok{))}
\end{Highlighting}
\end{Shaded}

Then, the map that included modeling of genotype errors was chosen as
the best one.

\begin{Shaded}
\begin{Highlighting}[]
\KeywordTok{plot_map_list}\NormalTok{(MAP.err)}
\end{Highlighting}
\end{Shaded}

\hypertarget{map-summary}{%
\subsection{Map summary}\label{map-summary}}

After building two or more maps, one may want to compare summary
statistics of those maps regarding the same chromosome, or even across
chromosomes. A brief comparison can be done using the function
\texttt{summary\_maps}, which generates a table containing these
statistics based on a list of \texttt{mappoly.map} objects:

\begin{Shaded}
\begin{Highlighting}[]
\NormalTok{knitr}\OperatorTok{::}\KeywordTok{kable}\NormalTok{(}\KeywordTok{summary_maps}\NormalTok{(MAPs))}
\end{Highlighting}
\end{Shaded}

\hypertarget{genotype-conditional-probabilities}{%
\section{Genotype conditional
probabilities}\label{genotype-conditional-probabilities}}

In order to use the genetic map in
\href{https://github.com/guilherme-pereira/QTLpoly}{QTLpoly}, one needs
to obtain the conditional probability of all possible 36 genotypes along
the 12 linkage groups for all individuals in the full-sib population.
Let us use the function \texttt{calc\_genoprob\_error}, which similarly
to \texttt{est\_full\_hmm\_with\_global\_error} allows the inclusion of
a global genotyping error:

\begin{Shaded}
\begin{Highlighting}[]
\NormalTok{genoprob.err <-}\StringTok{ }\KeywordTok{vector}\NormalTok{(}\StringTok{"list"}\NormalTok{, }\DecValTok{12}\NormalTok{)}
\ControlFlowTok{for}\NormalTok{(i }\ControlFlowTok{in} \DecValTok{1}\OperatorTok{:}\DecValTok{12}\NormalTok{)}
\NormalTok{genoprob.err[[i]] <-}\StringTok{ }\KeywordTok{calc_genoprob_error}\NormalTok{(}\DataTypeTok{input.map =}\NormalTok{ MAP.err[[i]], }\DataTypeTok{error =} \FloatTok{0.05}\NormalTok{)}
\end{Highlighting}
\end{Shaded}

Here, a global genotyping error of 5\% was used. Each position of any
object in the list \texttt{genoprob.err} contains two elements: an array
of dimensions
\(36 \times number \; of \; markers \times number \; of \; individuals\)
and the position of the markers in the maps in centimorgans. Let us
display the results for all linkage groups in individual 1:

\begin{Shaded}
\begin{Highlighting}[]
\NormalTok{ind <-}\StringTok{ }\DecValTok{1}
\NormalTok{op <-}\StringTok{ }\KeywordTok{par}\NormalTok{(}\DataTypeTok{mfrow =} \KeywordTok{c}\NormalTok{(}\DecValTok{3}\NormalTok{, }\DecValTok{4}\NormalTok{), }\DataTypeTok{pty =} \StringTok{"s"}\NormalTok{, }\DataTypeTok{mar=}\KeywordTok{c}\NormalTok{(}\DecValTok{1}\NormalTok{,}\DecValTok{1}\NormalTok{,}\DecValTok{1}\NormalTok{,}\DecValTok{1}\NormalTok{)) }
\ControlFlowTok{for}\NormalTok{(i }\ControlFlowTok{in} \DecValTok{1}\OperatorTok{:}\DecValTok{12}\NormalTok{)}
\NormalTok{\{}
\NormalTok{  d <-}\StringTok{ }\NormalTok{genoprob.err[[i]]}\OperatorTok{$}\NormalTok{map}
  \KeywordTok{image}\NormalTok{(}\KeywordTok{t}\NormalTok{(genoprob.err[[i]]}\OperatorTok{$}\NormalTok{probs[,,ind]),}
        \DataTypeTok{col=}\NormalTok{RColorBrewer}\OperatorTok{::}\KeywordTok{brewer.pal}\NormalTok{(}\DataTypeTok{n=}\DecValTok{9}\NormalTok{ , }\DataTypeTok{name =} \StringTok{"YlOrRd"}\NormalTok{),}
        \DataTypeTok{axes=}\OtherTok{FALSE}\NormalTok{,}
        \DataTypeTok{xlab =} \StringTok{"Markers"}\NormalTok{,}
        \DataTypeTok{ylab =} \StringTok{""}\NormalTok{,}
        \DataTypeTok{main =} \KeywordTok{paste}\NormalTok{(}\StringTok{"LG"}\NormalTok{, i))}
  \KeywordTok{axis}\NormalTok{(}\DataTypeTok{side =} \DecValTok{1}\NormalTok{, }\DataTypeTok{at =}\NormalTok{ d}\OperatorTok{/}\KeywordTok{max}\NormalTok{(d),}
       \DataTypeTok{labels =}\KeywordTok{rep}\NormalTok{(}\StringTok{""}\NormalTok{, }\KeywordTok{length}\NormalTok{(d)), }\DataTypeTok{las=}\DecValTok{2}\NormalTok{)}
\NormalTok{\}}
\KeywordTok{par}\NormalTok{(op)}
\end{Highlighting}
\end{Shaded}

In this figure, the x-axis represents the genetic map and the y-axis
represents the 36 possible genotypes in the full-sib population. The
color scale varies from dark purple (high probabilities) to light yellow
(low probabilities). With the conditional probabilities computed, it is
possible to use the object \texttt{genoprob.err} alongside with
phenotypic data as the input of the software
\href{https://github.com/guilherme-pereira/QTLpoly}{QTLpoly}, which is
an under development software to map multiple QTLs in full-sib families
of outcrossing autopolyploid species.

\hypertarget{obtaining-individual-haplotypes}{%
\section{Obtaining individual
haplotypes}\label{obtaining-individual-haplotypes}}

Once ready, the genotypic conditional probabilities can be used to
recover any individual haplotype given the map (details described in
Mollinari et al. (2020)). To generate this information, one may use the
function \texttt{calc\_homoprob} to account for the probabilities of
each homologous, in all map positions for each individual. For example,
let us view the homologous probabilities for chromosome 1 and individual
10:

\begin{Shaded}
\begin{Highlighting}[]
\NormalTok{homoprobs =}\StringTok{ }\KeywordTok{calc_homoprob}\NormalTok{(genoprob.err)}
\end{Highlighting}
\end{Shaded}

\begin{Shaded}
\begin{Highlighting}[]
\KeywordTok{plot}\NormalTok{(homoprobs, }\DataTypeTok{lg =} \DecValTok{1}\NormalTok{, }\DataTypeTok{ind =} \DecValTok{10}\NormalTok{)}
\end{Highlighting}
\end{Shaded}

Using this graphic, it is possible to identify regions of crossing-over
occurrence, represented by the inversion of probability magnitudes
between homologous from the same parent. It is also possible to view all
chromosomes at the same time for any individual by setting the parameter
\texttt{lg\ =\ "all"}. One may use this information to evaluate the
quality of the map and repeat some processes with modifications, if
necessary.

\hypertarget{evaluating-the-meiotic-process}{%
\section{Evaluating the meiotic
process}\label{evaluating-the-meiotic-process}}

MAPpoly also handles a function to evaluate the meiotic process that
guided gamete formation on the studied population. Given genotype
conditional probabilities, one may want to account for homologous
pairing probabilities and detect the occurrence of preferential pairing,
which is possible through the function
\texttt{calc\_prefpair\_profiles}:

\begin{Shaded}
\begin{Highlighting}[]
\NormalTok{prefpairs =}\StringTok{ }\KeywordTok{calc_prefpair_profiles}\NormalTok{(genoprob.err)}
\end{Highlighting}
\end{Shaded}

The function returns an object of class
\texttt{mappoly.prefpair.profiles}, which was saved as
\texttt{prefpairs}. This object handles all information necessary to
study pairing, such as the probability for each pairing configuration
(\(\psi\); see Mollinari and Garcia, 2019) inside each parent. For a
more user-friendly visualization of the results, one may want to look at
the \texttt{plot} output:

\begin{Shaded}
\begin{Highlighting}[]
\KeywordTok{plot}\NormalTok{(prefpairs)}
\CommentTok{#save.image("all.analysis.rda")}
\end{Highlighting}
\end{Shaded}

This graphic shows information about all pairing configurations and
their probabilities, the proportion of bivalent/multivalent pairing, and
also the p-value for preferential pairing test for all markers inside
each parent.

\hypertarget{references}{%
\section*{References}\label{references}}
\addcontentsline{toc}{section}{References}

\hypertarget{refs}{}
\leavevmode\hypertarget{ref-Mollinari2019}{}%
Mollinari, Marcelo, and Antonio Augusto Franco Garcia. 2019. ``Linkage
Analysis and Haplotype Phasing in Experimental Autopolyploid Populations
with High Ploidy Level Using Hidden Markov Models.'' \emph{G3: Genes,
Genomes, Genetics} 9 (10): 3297--3314.

\leavevmode\hypertarget{ref-Mollinari2020}{}%
Mollinari, Marcelo, Bode A Olukolu, Guilherme da S Pereira, Awais Khan,
Dorcus Gemenet, G Craig Yencho, and Zhao-Bang Zeng. 2020. ``Unraveling
the Hexaploid Sweetpotato Inheritance Using Ultra-Dense Multilocus
Mapping.'' \emph{G3: Genes, Genomes, Genetics} 10 (1): 281--92.

\leavevmode\hypertarget{ref-Pereira2020}{}%
Pereira, Guilherme da Silva, Dorcus C. Gemenet, Marcelo Mollinari, Bode
Olukolu, Federico Diaz, Veronica Mosquera, Wolfgang Gruneberg, Awais
Khan, Craig Yencho, and and Zhao-Bang Zeng. 2020. ``Multiple Qtl Mapping
in Autopolyploids: A Random-Effect Model Approach with Application in a
Hexaploid Sweetpotato Full-Sib Population.'' \emph{In Press}, no.
https://www.biorxiv.org/content/10.1101/622951v1.

\leavevmode\hypertarget{ref-Preedy2016}{}%
Preedy, K. F., and C. A. Hackett. 2016. ``A Rapid Marker Ordering
Approach for High-Density Genetic Linkage Maps in Experimental
Autotetraploid Populations Using Multidimensional Scaling.''
\emph{Theor. Appl. Genet.} 129 (11): 2117--32.
\url{https://doi.org/10.1007/s00122-016-2761-8}.

\end{document}
